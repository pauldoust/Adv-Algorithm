\documentclass[../report.tex]{subfiles}

%\usepackage{import}
%\usepackage[left=2cm, right=2cm, top=2.5cm]{geometry}
%\usepackage{algorithm}
%\usepackage{algorithmic}
%\usepackage[absolute,overlay]{textpos}
%\usepackage{graphicx}
%\usepackage[export]{adjustbox}
%\usepackage{caption}
%\usepackage{subcaption}


\begin{document}


The purpose of the algorithm is to get the cost of a path generated randomly from a set of node $S=\{0,...,n\}$, starting from node 0, with each node visited exactly once. The cost is the sum of the distance $d_{ij}$ between two successive nodes i and j. The distance for nodes that are not connected is set to -1.
\newline{} 
For a subset $S'\subseteq S$ of nodes not yet visited, starting from node j, we compute a probability distribution $P_{k}$, $k \in S'-\{j\}$, for the next possible nodes k based on the distance $d_{jk}$ such that the lower, the higher the probability. From the distribution P, the cumulative distribution P' is constructed using increasing probabilities $P_{k}$.  A random number r between 0 and 1 is picked from a uniform distribution. For r such that $P'_{k-1}< r <P'_{k}$, the node k is picked.


\[ sum = \sum_{k \in S'-\{j\}} d_{jk}, \quad {where} \quad d_{jk}!=-1 \]   
\[ P_{k} = (sum - d_{jk})/(sum*(n-1)), \quad \textit{where n number of node in S'-\{j\} connected to node j}\]
The division by (n-1) allows normalization so that $\sum_{k} P_{k}=1$,

\begin{figure}[ht]
\centering
\includegraphics[height=7cm]{random_algo.jpg}
\caption{Randomized algorithm \label{overflow}}
\end{figure}

\begin{subsection}{Pseudo code}\hfill

\begin{center}
	\colorbox[gray]{0.95}{
	\begin{minipage}{0.85\textwidth}
\begin{algorithm}[H]
\caption{Randomized Algorithm}
\begin{algorithmic} 
\REQUIRE{  }
   \STATE $d[nbNode][nbNode]$
   \STATE $S=\{1,...,n\}$
\ENSURE {}
\STATE sourcenode := 0
\STATE S' := S
\STATE path := []
\STATE $cost = 0$
\WHILE {$|S'|>1$}
   \STATE sum := 0
   \STATE nbnode := 0
   \FOR {node in S'-\{sourcenode\} with d[sourcenode][node] != -1}
      \STATE sum = sum + d[sourcenode][node]
      \STATE nbnode= nbnode + 1
   \ENDFOR   
   \FOR {k in S'-\{sourcenode\} with d[sourcenode][node] != -1}
      \STATE $P(k) = (sum - d[sourcenode][k])/sum/(nbnode-1)$
   \ENDFOR
   \STATE sort(P)
   \STATE P' := cumulative sum of elements of P
   \STATE r := random(0,1)
   \STATE search k s.t. $P(k-1)<r<P(k)$
   \STATE $cost := cost + d[sourcenode][k]$
   \STATE add k to path
   \STATE remove k from S'
   \STATE $startnode := k$   
\ENDWHILE
\STATE $cost := cost + d[startnode][0]$
\RETURN $cost$, $path$
\end{algorithmic}
\end{algorithm}
\end{minipage}}
\end{center}

\end{subsection}

\begin{subsection}{Complexity}
The algorithm is proceeding from top to down. For each node source (n loops) we compute the probability table of the possible next nodes using operations of the order $O(1)$. The table is sorted in increasing order using Python sort function whose complexity is $O(n log n$). The search of the node whose probability matches the random number is O(log n) using divide and conquer algorithm.
Consequently, the time complexity for the proposed version of randomized algorithm is $O(n^2 log n)$.
For each source node, we compute a probability table of at most n elements. The space complexity is O(n).
\end{subsection}

\begin{subsection}{Performances}
The algorithm provides an estimate of the cost. By running it multiple times we can attempt to get best result.
The algorithm is tested on a set of problems generated randomly with different vertex numbers and different sparsity levels.
Figure ~\ref{fig:randomgraphsize} shows the evolution of the algorithm running time with the number of cities. We can observe that the curves for symetric and asymetric problems follow the theorical complexity of $O(n^2 log n)$. \\
Figure ~\ref{fig:randomsparsity} shows the evolution of the running time with respect to the sparsity level. The running time increases when the connectivity increases (therefore when the sparsity decreases) as we have to compute the probabilities on more vertices. When the sparsity is high there is hardly any choice to make.
The implementation makes no difference symetric and asymetric problems. So the similarity of results between both types of problems was expected.

\begin{figure}[H]
\centering
\begin{subfigure}{.5\textwidth}
\includegraphics[height=4.8cm,valign=t]{perf_random_graphsize.jpg}
\caption{Impact of the number of vertices \label{fig:randomgraphsize}}
\end{subfigure}%
\begin{subfigure}{.5\textwidth}
\includegraphics[height=4.8cm,valign=t]{perf_random_sparsity.jpg}
\caption{Impact of the sparsity \label{fig:randomsparsity}}
\end{subfigure}%
\caption{Performance study \label{fig:randomperf}}
\end{figure}

The algorithm is tested on a set of problems from the TSLIB. Results are shown in section 11. The random algorithm is really reaching the optimal solution even after running several times the same problem.


\end{subsection}
\end{document}


